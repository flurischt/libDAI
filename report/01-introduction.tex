
\section{Introduction}\label{sec:intro}

Do not start the introduction with the abstract or a slightly modified
version. It follows a possible structure of the introduction. 
Note that the structure can be modified, but the
content should be the same. Introduction and abstract should fill at most the first page, better less.

\mypar{Motivation} The first task is to motivate what you do.  You can
start general and zoom in one the specific problem you consider.  In
the process you should have explained to the reader: what you are doing,
why you are doing, why it is important (order is usually reversed).

For example, if my result is the fastest DFT implementation ever, one
could roughly go as follows. First explain why the DFT is important
(used everywhere with a few examples) and why performance matters (large datasets,
realtime). Then explain that fast implementations are very hard and
expensive to get (memory hierarchy, vector, parallel). 

Now you state what you do in this paper. In our example: 
presenting a DFT implementation that is
faster for some sizes than all the other ones.

\mypar{Related work} Next, you have to give a brief overview of
related work. For a paper like this, anywhere between 2 and 8
references. Briefly explain what they do. In the end contrast to what
you do to make now precisely clear what your contribution is.