\section{Background}\label{sec:background}

%Give a short, self-contained summary of necessary
%background information. For example, assume you present an
%implementation of FFT algorithms. You could organize into DFT
%definition, FFTs considered, and cost analysis. The goal of the
%background section is to make the paper self-contained for an audience
%as large as possible. As in every section
%you start with a very brief overview of the section. Here it could be as follows: In this section 
%we formally define the discrete Fourier transform, introduce the algorithms we use
%and perform a cost analysis.

\mypar{Belief Propagation}
Belief Propagation (BP) is a technique in the domain of machine learning to perform probabilistic inference on graphical models, out of which Bayesian Networks and Markov Random Fields are the most well known. The standard version of BP is designed to deal with factor graphs, a bipartite graph that can represent Bayesian Networks or Markov Random Fields. Each node of the factor graphs represents the state of a random variable. Through message passing the state of the nodes is inferred. BP, also known as sum-product algorithm, computes the message sent from node $v_i$ to $v_j$ using equation \ref{eqn:bp_message}.
\begin{equation}                                                            
\label{eqn:bp_message}
m_{ij}(x_c) = \sum_{x_d \in X}\phi_i(x_d)\psi_{ij}(x_d,x_c)\prod_{k \in N(i) \ j}m_{ki}(x_d)
\end{equation}

\textbf{TODO: describe formula, add loopy etc.} \\


\mypar{Top-N Recommendation}
Top-N Recommendation is the problem where for a user $i$ one generates a list of n recommended objects, in our case movies, given a list of ratings by user $i$ and others. Ha et al. \cite{Ha:2012:TRT:2396761.2398636} propose a method for top-n recommendation using belief propagation.

\mypar{Cost Analysis}
..

%\mypar{Discrete Fourier Transform}
%Precisely define the transform so I understand it even if I have never
%seen it before.
%
%\mypar{Fast Fourier Transforms}
%Explain the algorithm you use.
%
%\mypar{Cost Analysis}
%First define you cost measure (what you count) and then compute the
%cost. Ideally precisely, at least asymptotically. In the latter case you will need to instrument your code to count
%the operations so you can create a performance plot.
%
%Also state what is
%known about the complexity (asymptotic usually) 
%about your problem (including citations).